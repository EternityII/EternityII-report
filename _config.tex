% !TeX root = main.tex

\usepackage[utf8]{inputenc}
\usepackage[french]{babel}
\usepackage[T1]{fontenc}
%====================== PACKAGES ======================
\usepackage[backend=biber,sorting=none]{biblatex}
\usepackage[mode=buildnew]{standalone}
%pour gérer les positionnement d'images
\usepackage{float}
\usepackage{amsmath,amssymb,amsthm}
\usepackage{graphicx}
\usepackage[colorinlistoftodos]{todonotes}
\usepackage[toc,page]{appendix}
\renewcommand\appendixtocname{Annexes}
\renewcommand\appendixpagename{Annexes}
\usepackage{url}
%pour les informations sur un document compilé en PDF et l\section{}es liens externes / internes
\usepackage{hyperref}
%pour la mise en page des tableaux
\usepackage{array}
\usepackage{tabularx}
\usepackage[thinlines]{easytable}
%pour utiliser \floatbarrier
%\usepackage{placeins}
%\usepackage{floatrow}
%espacement entre les lignes
\usepackage{setspace}
%modifier la mise en page de l'abstract
\usepackage{fancyhdr}
\usepackage{abstract}
%police et mise en page (marges) du document

\usepackage{pdfpages}

\usepackage{csquotes}
%Code source
\usepackage{listingsutf8}
\lstset{basicstyle=\ttfamily,
	keywordstyle=\color{blue}\ttfamily,
	stringstyle=\color{red}\ttfamily,
	commentstyle=\color{green}\ttfamily,
	morecomment=[l][\color{magenta}]{\#},
    extendedchars=false,
    inputencoding=utf8,
    numberstyle=\footnotesize,
    numbers=left,
    frame=L,
	breaklines=true,
	postbreak=\raisebox{0ex}[0ex][0ex]{\ensuremath{\color{red}\hookrightarrow\space}}
}

\usepackage[top=2cm, bottom=2cm, left=2cm, right=2cm]{geometry}
%Pour les galerie d'images
\usepackage{subfig}

% Glossaire
\usepackage[xindy]{glossaries}	% Ensures that all acronyms are defined once
\makeglossaries

\pagestyle{fancy}
\headheight=14.62pt

\setlength{\parindent}{4em}
\setlength{\parskip}{1em}

%% Mise en page des théorèmes, lemme etc...
\theoremstyle{plain}
\newtheorem{thm}{Théorème}[section]
\newtheorem{lemme}[thm]{Lemme}
\newtheorem{prop}[thm]{Proposition}
\newtheorem*{cor}{Corollaire}
\theoremstyle{definition}
\newtheorem{defn}{Définition}[section]
\newtheorem*{defn*}{Définition}
\newtheorem{law}{Loi}
\newtheorem*{law*}{Loi}
\newtheorem{conj}{Conjecture}[section]
\newtheorem{exmp}{Exemple}[section]
\newtheorem*{exmp*}{Exemple}
\theoremstyle{remark}
\newtheorem*{rem}{Remarque}
\newtheorem*{note}{Note}
\newtheorem{case}{Cas particulier}

\renewcommand{\floatpagefraction}{0.95}
\renewcommand{\textfraction}{0.05}