\section{Etat de l'art}

Un grand nombre de méthodes ont été mis en place afin de résoudre ce problème.

Il serait trop long de présenter et décrire les différents méthodes mise en place car il requièrent une certaine connaissance dans les domaines auxquels ils sont appliqués. Malgré tout, Les différentes approches seront notés ici à titre informatif.

Pour commencer, il existe plusieurs solveurs graphiques afin de pouvoir résoudre le puzzle manuellement ou assisté par l'ordinateur. Certains d'entre eux permettent même l'import export de la progression actuelle.

\begin{itemize}
	\item E2\_manual : \url{https://sourceforge.net/projects/e2manual/}(anglais)
	\item E2Lab : \url{http://eternityii.free.fr/}(anglais)
\end{itemize}

Ensuite, il existe un très grand nombre de solveurs bruteforce plus au moins rapides. Certains d'entre eux peuvent agréger plusieurs machines afin d'augmenter la puissance de calcul via le réseau.

Ce problème à été abordé de façon très varié au niveau théorique et applicatif.

Par exemple, Eternity à été adopté sous forme de graphe \cite{patey2010eternity} ou de sous forme de contraintes \cite{benoist2008programmation}.

On note aussi un procédé intéressent de résolution grâce à une approche nommée \textbf{SAT} (satisfaisabilité booléenne) s'appuyant sur la logique propositionnelle \cite{cuvillierresolution} \cite{heule2009solving}.

--- [résumé]

Nombreux sont ceux qui ont essayé de résoudre le problème... plusieurs moyens ont été mis en place, grâce au solveur SAT (solveur de satisfiabilité booléenne dont le fonctionnement est assez complexe pour ne pas être abordé ici), par une approche graphe ou encore par bruteforce.

Il est estimé que actuellement, l'approche la plus performante est la résolution par bruteforce, car c'est elle qui est la plus rapide.

[à développer]