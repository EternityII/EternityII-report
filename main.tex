\documentclass{article}
% !TeX root = main.tex

\usepackage[utf8]{inputenc}
\usepackage[french]{babel}
\usepackage[T1]{fontenc}
%====================== PACKAGES ======================
\usepackage[backend=biber,sorting=none]{biblatex}
\usepackage[mode=buildnew]{standalone}
%pour gérer les positionnement d'images
\usepackage{float}
\usepackage{amsmath,amssymb,amsthm}
\usepackage{graphicx}
\usepackage[colorinlistoftodos]{todonotes}
\usepackage[toc,page]{appendix}
\renewcommand\appendixtocname{Annexes}
\renewcommand\appendixpagename{Annexes}
\usepackage{url}
%pour les informations sur un document compilé en PDF et l\section{}es liens externes / internes
\usepackage{hyperref}
%pour la mise en page des tableaux
\usepackage{array}
\usepackage{tabularx}
\usepackage[thinlines]{easytable}
%pour utiliser \floatbarrier
%\usepackage{placeins}
%\usepackage{floatrow}
%espacement entre les lignes
\usepackage{setspace}
%modifier la mise en page de l'abstract
\usepackage{fancyhdr}
\usepackage{abstract}
%police et mise en page (marges) du document

\usepackage{pdfpages}

\usepackage{csquotes}
%Code source
\usepackage{listingsutf8}
\lstset{basicstyle=\ttfamily,
	keywordstyle=\color{blue}\ttfamily,
	stringstyle=\color{red}\ttfamily,
	commentstyle=\color{green}\ttfamily,
	morecomment=[l][\color{magenta}]{\#},
    extendedchars=false,
    inputencoding=utf8,
    numberstyle=\footnotesize,
    numbers=left,
    frame=L,
	breaklines=true,
	postbreak=\raisebox{0ex}[0ex][0ex]{\ensuremath{\color{red}\hookrightarrow\space}}
}

\usepackage[top=2cm, bottom=2cm, left=2cm, right=2cm]{geometry}
%Pour les galerie d'images
\usepackage{subfig}

% Glossaire
\usepackage[xindy]{glossaries}	% Ensures that all acronyms are defined once
\makeglossaries

\pagestyle{fancy}
\headheight=14.62pt

\setlength{\parindent}{4em}
\setlength{\parskip}{1em}

%% Mise en page des théorèmes, lemme etc...
\theoremstyle{plain}
\newtheorem{thm}{Théorème}[section]
\newtheorem{lemme}[thm]{Lemme}
\newtheorem{prop}[thm]{Proposition}
\newtheorem*{cor}{Corollaire}
\theoremstyle{definition}
\newtheorem{defn}{Définition}[section]
\newtheorem*{defn*}{Définition}
\newtheorem{law}{Loi}
\newtheorem*{law*}{Loi}
\newtheorem{conj}{Conjecture}[section]
\newtheorem{exmp}{Exemple}[section]
\newtheorem*{exmp*}{Exemple}
\theoremstyle{remark}
\newtheorem*{rem}{Remarque}
\newtheorem*{note}{Note}
\newtheorem{case}{Cas particulier}

\renewcommand{\floatpagefraction}{0.95}
\renewcommand{\textfraction}{0.05}

%====================== INFORMATION PDF======================

\hypersetup{												% Information sur le document
	pdfauthor = {Fati CHEN},								% Auteurs
	pdftitle = {Ouvertures et Finales d'Eternity II},		% Titre du document
	pdfsubject = {Rapport de Projet},						% Sujet
	pdfkeywords = {Eternity II, rapport de projet, LIRMM},	% Mots-clefs
	pdfstartview={FitH}										% ajuste la page à la largueur de l'écran
	pdfcreator = {MikTeX},									% Logiciel qui a crée le document
	pdfproducer = {Eternithug}}								% Société avec produit le logiciel


%======================== DEBUT DU DOCUMENT ========================
\begin{document}
	
	%__ Listes
	\renewcommand{\labelitemi}{$\bullet$}
	\renewcommand{\labelitemii}{$\cdot$}
	\renewcommand{\labelitemiii}{$\diamond$}
	\renewcommand{\labelitemiv}{$\ast$}
	
	%__ régler l'espacement entre les lignes
	\newcommand{\HRule}{\rule{\linewidth}{0.5mm}}
	
	%========================= Debut du texte =========================
	
	% !TeX root = main.tex

\pagenumbering{Alph}

\begin{titlepage}
	\begin{center}
	
	% Upper part of the page. The '~' is needed because only works if a paragraph has started.
	%\includegraphics[width=0.35\textwidth]{./logo}~\\[1cm]
	
	\textsc{\LARGE Faculté des Sciences\\
		Université de Montpellier\\
		Lirmm\\} \ \\[1.5cm]
	
	\textsc{\Large }\\[0.5cm]
	
	% Title
	\HRule \\[0.4cm]
	
	{\huge \bfseries Rapport de stage\\
	Ouvertures et finales d'Eternity II\\[0.4cm] }
	
	\HRule \\[1.5cm]
	
	% Author and supervisor
	\begin{minipage}{0.4\textwidth}
		\begin{flushleft} \large
			\emph{Auteurs:}\\
				Fati \textsc{Chen}\\
		\end{flushleft}
	\end{minipage}
	\begin{minipage}{0.4\textwidth}
		\begin{flushright} \large
			\emph{Référent:} \\
				Eric \textsc{Bourreau}
		\end{flushright}
	\end{minipage}
	
	\vfill
	\thispagestyle{empty}

	% Bottom of the page
	{\large \today}
	\end{center}
\end{titlepage}

\pagenumbering{arabic}
	
	\section{Introduction}
	
	\section{Eternity II}
	\subsection{Le jeu}
	
	Eternity II est un jeu sorti en 2008 qui repose sur un principe assez simple, c'est un puzzle de 16 par 16 qu'il faut réassembler. Il est composé de 256 pièces carrés, qui ont, sur chaque arête une couleur donnée. [images tt ca tt ca]. Afin de pouvoir assembler le puzzle, il suffit placer les pièces de façon à ce que les faces adjacentes soient de même couleur. Comme un puzzle classique, il y a des pièces de coin et de bord. Ceux-ci sont reconnaissables car ils possèdent une ou deux arêtes grises. Par contre, la où ca devient complexe, c'est qu'une pièce n'as pas une place prédéterminée (comme dans un puzzle), c'est à dire qu'elle peux se situer n'importe où sur le plateau. 
	
	\subsection{Le défi}
	
	A ce jour, personne n'a réussi à résoudre ce puzzle (même grâce à l'aide de supercalculateurs) malgré les différentes stratégies mise en place. Pourquoi ? Car derrière ce jeu anodin se cache l'un des plus grand problème du monde actuel : les problèmes NP-difficiles. Ceux-ci sont fait de tel sorte que même en connaissant leur structure ou fonctionnement, il est pratiquement impossible d'en déduire un algorithme de résolution. L'une des solutions les plus fiables à ce jour est de tester tout les cas possible (qui est évidemment très important).
	
	\begin{exmp}
		Le nombre de combinaisons pour Eternity II s'élève à $10^{545}$, c'est à dire environ $10^{450}$ fois le nombre d'atomes dans l'univers connu (estimé à au plus $10^{80}$) !!!
	\end{exmp}
	
	\subsection{La recette secrète d'Eternity II}
	
	Pour rendre ce problème combinatoire, il est nécessaire de respecter plusieurs conditions.
	
	\begin{description}
		\item[Chaque pièce est unique] l'unicité des pièces est importante, car si une pièce est en double, cela veux dire que la pièce peux être placée à deux endroits différents (ce qui simplifie le problème)
		\item[Le ratio de nombre de couleurs par nombre de pièces est calculé] [joindre graphique tt ca tt ca] : il faut qu'il y ait assez de couleur pour que chaque pièce soit unique, mais pas assez pour que l'on puisse déterminer les pièces adjacentes
		
		\begin{exmp}
			Supposons qu'il y ait trop de couleurs. Cela veux dire qu'une pièce à peu de voisins ( car chaque pièce est unique, par conséquent les couleurs sont distribués uniformément à travers les pièces), si la pièce à très peu de voisins, je peux déduire des groupement de pièces assez facilement.
			
			Donc je simplifie mon problème.
		\end{exmp}		
	\end{description}
	
	\subsection{Etat de l'art}
	
	Nombreux sont ceux qui ont essayé de résoudre le problème... plusieurs moyens ont été mis en place, grâce au solveur SAT (solveur de satisfiabilité booléenne dont le fonctionnement est assez complexe pour ne pas être abordé ici), par une approche graphe ou encore par bruteforce.
	
	Il est estimé que actuellement, l'approche la plus performante est la résolution par bruteforce, car c'est elle qui est la plus rapide.
	
	\section{Sujet / problématique}
	
	Le but du stage est donc de déterminer si, malgré les dires, on ne pourrait pas trouver une méthode de résolution plus efficace que la bruteforce, qui repose sur une grande quantité d'information pré-calculées, cette méthode est appellée smartforce par la suite.
	
	Ces informations pré-calculées serviront différentes causes, mais deux objectifs principaux peuvent en être explicités :
	
	\begin{itemize}
		\item les ouvertures
		\item les finales
	\end{itemize}
	
	Afin de comprendre le principe des ouvertures et des finales, il est important de pouvoir se représenter un arbre de possibilités où chaque branche de l'arbre est une combinaison spécifique. [là faut une bonne image explicite]
	
	\subsection{Ouvertures}
	
	Le but des ouvertures, est de pré-calculer jusqu'à un certain niveau toutes les combinaisons possibles, et de les stocker. Cela permet par la suite, de pouvoir paralléliser les calculs afin de tester plusieurs sous instances du problème
	
	\subsection{Finales}
	
	De la même façon, pré-calculer les fins possibles permet de connaitre plus rapidement si la combinaison actuelle est possible ou non.
	
	
	\subsection{Difficultés préliminaires}
	
	afin de résoudre le problème, la principale difficulté c'est que le jeu de base est trop complexe et ne permet pas de déterminer si l'approche actuelle est adapté. Par conséquent, on utilisera des instances d'Eternity II, qui sont des plateaux de plus petite taille (4x4, 5x5 ...) qui ont les mêmes propriétés que le jeu de base.
	
	\section{Approche}

	Dans cette partie, nous expliquerons quels sont les différents outils et stratégies mis en place pour permettre la résolution du problème.
	Dans un premier temps nous avons eu besoin de mettre en place une valeur étalon, qui nous permet de savoir si les differentes méthodes de smartforce sont bonnes ou pas. Cette valeur étalon est un programme de bruteforce qui nous fournit le nombre total de noeuds, le nombre de solution, le nombre de noeuds à la premiere solution et les tous les timers correspondant.
	
	\subsection{Bruteforce}
	
	\subsection{Corolles}
	
	\subsection{CaPi}
	
	\subsection{BoCo}
	
	\subsection{BoCoDiag}
	
	\section{Manuel d'utilisation}
	
	\section{Manuel Technique}

\end{document}