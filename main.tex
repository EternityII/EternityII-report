
\documentclass{article}
\usepackage[utf8]{inputenc}
\usepackage[french]{babel}
\usepackage[T1]{fontenc}
%====================== PACKAGES ======================
%pour gérer les positionnement d'images
\usepackage{float}
\usepackage{amsmath,amssymb,amsthm}
\usepackage{graphicx}
\usepackage[colorinlistoftodos]{todonotes}
\usepackage{url}
%pour les informations sur un document compilé en PDF et l\section{}es liens externes / internes
\usepackage{hyperref}
%pour la mise en page des tableaux
\usepackage{array}
\usepackage{tabularx}
\usepackage[thinlines]{easytable}
%pour utiliser \floatbarrier
%\usepackage{placeins}
%\usepackage{floatrow}
%espacement entre les lignes
\usepackage{setspace}
%modifier la mise en page de l'abstract
\usepackage{fancyhdr}
\usepackage{abstract}
%police et mise en page (marges) du document

\usepackage{pdfpages}

\usepackage{csquotes}
%Code source
\usepackage{listingsutf8}

\usepackage[top=2cm, bottom=2cm, left=2cm, right=2cm]{geometry}
%Pour les galerie d'images
\usepackage{subfig}

% Glossaire
\usepackage[xindy]{glossaries}	% Ensures that all acronyms are defined once
\makeglossaries

\pagestyle{fancy}
\headheight=14.62pt

%% Mise en page des théorèmes, lemme etc...
\theoremstyle{plain}
\newtheorem{thm}{Théorème}[section]
\newtheorem{lemme}[thm]{Lemme}
\newtheorem{prop}[thm]{Proposition}
\newtheorem*{cor}{Corollaire}
\theoremstyle{definition}
\newtheorem{defn}{Définition}[section]
\newtheorem{conj}{Conjecture}[section]
\newtheorem{exmp}{Exemple}[section]
\theoremstyle{remark}
\newtheorem*{rem}{Remarque}
\newtheorem*{note}{Note}
\newtheorem{case}{Cas particulier}

\renewcommand{\floatpagefraction}{0.95}
\renewcommand{\textfraction}{0.05}

%====================== INFORMATION PDF======================

\hypersetup{												% Information sur le document
	pdfauthor = {Fati CHEN},								% Auteurs
	pdftitle = {Ouvertures et Finales d'Eternity II},		% Titre du document
	pdfsubject = {Rapport de Projet},						% Sujet
	pdfkeywords = {Eternity II, rapport de projet, LIRMM},	% Mots-clefs
	pdfstartview={FitH}										% ajuste la page à la largueur de l'écran
	pdfcreator = {MikTeX},									% Logiciel qui a crée le document
	pdfproducer = {Eternithug}}								% Société avec produit le logiciel


%======================== DEBUT DU DOCUMENT ========================
\begin{document}
	
	%__ Listes
	\renewcommand{\labelitemi}{$\bullet$}
	\renewcommand{\labelitemii}{$\cdot$}
	\renewcommand{\labelitemiii}{$\diamond$}
	\renewcommand{\labelitemiv}{$\ast$}
	
	%__ régler l'espacement entre les lignes
	\newcommand{\HRule}{\rule{\linewidth}{0.5mm}}
	
	%========================= Debut du texte =========================
	
	\pagenumbering{Alph}

\begin{titlepage}
	\begin{center}
	
	% Upper part of the page. The '~' is needed because only works if a paragraph has started.
	%\includegraphics[width=0.35\textwidth]{./logo}~\\[1cm]
	
	\textsc{\LARGE Faculté des Sciences\\
		Université de Montpellier\\
		Lirmm\\} \ \\[1.5cm]
	
	\textsc{\Large }\\[0.5cm]
	
	% Title
	\HRule \\[0.4cm]
	
	{\huge \bfseries Rapport de stage\\
	Ouvertures et finales d'Eternity II\\[0.4cm] }
	
	\HRule \\[1.5cm]
	
	% Author and supervisor
	\begin{minipage}{0.4\textwidth}
		\begin{flushleft} \large
			\emph{Auteurs:}\\
				Fati \textsc{Chen}\\
		\end{flushleft}
	\end{minipage}
	\begin{minipage}{0.4\textwidth}
		\begin{flushright} \large
			\emph{Référent:} \\
				Eric \textsc{Bourreau}
		\end{flushright}
	\end{minipage}
	
	\vfill
	\thispagestyle{empty}

	% Bottom of the page
	{\large \today}
	\end{center}
\end{titlepage}

\pagenumbering{arabic}
	
	\section{Introduction}
	
	Les puzzles et casses-têtes nous ont toujours passionnés, pour faire passer le temps ou pour se mettre des défis. EternityII est un de ces jeux où le principe peux être compris par tous, mais pourtant sa résolution est extrêmement complexe. Ce genre de paradigme est à l'heure actuelle l'un des problèmes mathématiques qui régit notre monde, car la plupart des systèmes informatiques et méthodes de chiffrement reposent sur ce genre de problème (simple à faire mais pourtant trouver la solution ne l'est pas).
	
	EternityII n'est résoluble à l'heure actuelle qu'en testant toutes les combinaisons (bruteforce). Ce qui nous fait poser une question importante, comment, avec l'augmentation exponentielle des données et des nouvelles technologies, sommes nous réduit à utiliser une méthode aussi simple.
	Par extension, est-il plus efficace d'accumuler des données afin de le résoudre plutôt qu'essayer d'accélérer la résolution basique. 
	
	Dans un premier temps, nous verrons les origines du jeu, la difficulté à laquelle nous sommes confrontés et l'état de l'art des méthodes de résolutions.
	
	Ensuite, nous présenterons la problématique, ce qui à déjà tout au long de l'année et l'approche initiale du problème.
	
	Pour conclure, les résultats et réflexions qui peuvent en être tirés.
	
	Par ailleurs, ce compte rendu comporte un manuel d'utilisation et un manuel technique fourni, car les application développées, ou tout du moins leur logique est destinée à être réutilisés ou améliorés.
	
	\section{Eternity II}
	\subsection{Les origines}
	--- [réel]
	
	EternityII est le fier successeur de Eternity.
	
	La première version [image](photo de la première version), sortie en 1999, était composée de 159 pièces de différentes formes, cependant ces formes peuvent être décomposés en formes de trianges equilatéraux (ou leur moitié) [image](photo d'exemple d'une piece) qui devaient être placés sur un plateau héxagonal. Son point faible se trouvaient dans la disposition de ces pièces sur le plateau : il était possible de précalculer des régions, puis de les comparer entre eux afin d'en dégager une solution.
	De cette facon, le puzzle fut résolu en à peine un an (contrairement aux 3 ans prévus par le créateur), par deux mathématiciens, qui ont ainsi empoché la récompense s'élevant à $1000000\pounds$.
	
	Après cet "échec", Christopher Monckton, le créateur d'Eternity, décida en 2008 de sortir une deuxième version, bien plus complexe avec à la clé $2000000\textdollar$ pour celui qui arriverai à le résoudre au bout de deux ans.
	
	[image](eternityII boite)
	
	C'est un puzzle de 16 par 16 qui sort sous le nom d'EternityII. Ce puzzle est composé de 256 pièces carrés, qui ont chacune 4 faces colorés.
	
	Ces pièces peuvent être classés en trois catégories suivant le nombre de faces grises qu'elles possèdent :
	
	[image](le triplet gagnant)
	
	\begin{description}
		\item [pièce de coin] 2 faces grises
		\item [pièce de bord] 1 faces grise
		\item [pièce d'intérieur] toutes les faces de couleur
	\end{description}
	
	Les pièces ne possèdent pas de formes comme dans un puzzle classique. Afin de les faire correspondre l'une avec l'autre, il est nécessaire que les faces adjacentes de chaque pièce voisine soient de la même couleur. Par conséquent, la pièce peut être placée n'importe ou sur le plateau car son placement dépends des couleurs des pièces d'à côté, de plus, les pièces n'ont pas d'orientation prédéterminés (elle peut être rotationné).
	
	Pour résumer, la plupart des pièces peuvent être posés n'importe où sur le plateau à différentes rotation car la position dépends entièrement des pièces adjacentes posés auparavant. 
	
	Effectivement, comme leur nom l'indiquent, les pièces de coins sont les seules à pouvoir être posés dans les coins du plateau, c'est aussi valable pour les pièces de bord qui ne peuvent être placés que sur les bords du plateau.
	
	
	--- [résumé]
	
	Eternity II est un jeu sorti en 2008 qui repose sur un principe assez simple, c'est un puzzle de 16 par 16 qu'il faut réassembler. Il est composé de 256 pièces carrés, qui ont, sur chaque arête une couleur donnée. [images tt ca tt ca]. Afin de pouvoir assembler le puzzle, il suffit placer les pièces de façon à ce que les faces adjacentes soient de même couleur. Comme un puzzle classique, il y a des pièces de coin et de bord. Ceux-ci sont reconnaissables car ils possèdent une ou deux arêtes grises. Par contre, la où ca devient complexe, c'est qu'une pièce n'as pas une place prédéterminée (comme dans un puzzle), c'est à dire qu'elle peux se situer n'importe où sur le plateau. 
	
	\subsection{Le défi}
	
	A ce jour, personne n'a réussi à résoudre ce puzzle (même grâce à l'aide de supercalculateurs) malgré les différentes stratégies mise en place. Pourquoi ? Car derrière ce jeu anodin se cache l'un des plus grand problème du monde actuel : les problèmes NP-difficiles. Ceux-ci sont fait de tel sorte que même en connaissant leur structure ou fonctionnement, il est pratiquement impossible d'en déduire un algorithme de résolution. L'une des solutions les plus fiables à ce jour est de tester tout les cas possible (qui est évidemment très important).
	
	\begin{exmp}
		Le nombre de combinaisons pour Eternity II s'élève à $10^{545}$, c'est à dire environ $10^{450}$ fois le nombre d'atomes dans l'univers connu (estimé à au plus $10^{80}$) !!!
	\end{exmp}
	
	\subsection{La recette secrète d'Eternity II}
	
	Pour rendre ce problème combinatoire, il est nécessaire de respecter plusieurs conditions.
	
	\begin{description}
		\item[Chaque pièce est unique] l'unicité des pièces est importante, car si une pièce est en double, cela veux dire que la pièce peux être placée à deux endroits différents (ce qui simplifie le problème)
		\item[Le ratio de nombre de couleurs par nombre de pièces est calculé] [joindre graphique tt ca tt ca] : il faut qu'il y ait assez de couleur pour que chaque pièce soit unique, mais pas assez pour que l'on puisse déterminer les pièces adjacentes
		
		\begin{exmp}
			Supposons qu'il y ait trop de couleurs. Cela veux dire qu'une pièce à peu de voisins ( car chaque pièce est unique, par conséquent les couleurs sont distribués uniformément à travers les pièces), si la pièce à très peu de voisins, je peux déduire des groupement de pièces assez facilement.
			
			Donc je simplifie mon problème.
		\end{exmp}		
	\end{description}
	
	\subsection{Etat de l'art}
	
	Nombreux sont ceux qui ont essayé de résoudre le problème... plusieurs moyens ont été mis en place, grâce au solveur SAT (solveur de satisfiabilité booléenne dont le fonctionnement est assez complexe pour ne pas être abordé ici), par une approche graphe ou encore par bruteforce.
	
	Il est estimé que actuellement, l'approche la plus performante est la résolution par bruteforce, car c'est elle qui est la plus rapide.
	
	[à développer]
	
	\section{Sujet / problématique}
	
	Le but du stage est donc de déterminer si, malgré les dires, on ne pourrait pas trouver une méthode de résolution plus efficace que la bruteforce, qui repose sur une grande quantité d'information pré-calculées, cette méthode est appellée smartforce par la suite.
	
	Ces informations pré-calculées serviront différentes causes, mais deux objectifs principaux peuvent en être explicités :
	
	\begin{itemize}
		\item les ouvertures
		\item les finales
	\end{itemize}
	
	Afin de comprendre le principe des ouvertures et des finales, il est important de pouvoir se représenter un arbre de possibilités où chaque branche de l'arbre est une combinaison spécifique. [là faut une bonne image explicite]
	
	\subsection{Ouvertures}
	
	Le but des ouvertures, est de pré-calculer jusqu'à un certain niveau toutes les combinaisons possibles, et de les stocker. Cela permet par la suite, de pouvoir paralléliser les calculs afin de tester plusieurs sous instances du problème
	
	\subsection{Finales}
	
	De la même façon, pré-calculer les fins possibles permet de connaitre plus rapidement si la combinaison actuelle est possible ou non.
	
	
	\subsection{Difficultés préliminaires}
	
	afin de résoudre le problème, la principale difficulté c'est que le jeu de base est trop complexe et ne permet pas de déterminer si l'approche actuelle est adapté. Par conséquent, on utilisera des instances d'Eternity II, qui sont des plateaux de plus petite taille (4x4, 5x5 ...) qui ont les mêmes propriétés que le jeu de base.
	
	\section{Approche}

	Dans cette partie, nous expliquerons quels sont les différents outils et stratégies mis en place pour permettre la résolution du problème.
	Dans un premier temps nous avons eu besoin de mettre en place une valeur étalon, qui nous permet de savoir si les differentes méthodes de smartforce sont bonnes ou pas. Cette valeur étalon est un programme de bruteforce qui nous fournit le nombre total de noeuds, le nombre de solution, le nombre de noeuds à la premiere solution et les tous les timers correspondant.
	
	\subsection{Bruteforce}
	
	Afin de pouvoir partir sur de bonnes bases, plusieurs différents méthodes de parcours (quel chemin prendre pour résoudre mon plateau) ont été utilisé, afin de voir quel parcours est le plus performant pour la résolution brute. En sachant que le nombre de noeuds/sec est la même, l'unité de mesure est le nombre de noeuds.
	
	Les differents types de parcours sont :
	
	\begin{description}
		\item[rowscan] on pose les pieces en lignes horizontales sur le plateau
		\item[diagonal] on pose les pieces en diagonal
		\item[spiral in] on dispose les pièces en spirale en partant de l'exterieur [image]
		\item[spiral out] idem que spiral in mais en partant de l'interieur vers l'exterieur
	\end{description}
	
	Ces différents types de parcours ont été testés sur plusieurs instances de taille variable. 
	
	\subsection{Smartforce}
	
	Une fois la valeur étalon fixée, il est maintenant facile de mettre en place un autre approche du problème qui à pour principe de cumuler une grande quantité de donnée pour faire face au nombre exponentiel de possibilités.
	
	Les différents types de données (nommés modèles) sont comme différents points de vues du problème. Ils sont plus ou moins utiles, mais la force réside dans leur union. Mais surtout, ils permettent de mettre en place le concept d'ouvertures et finales.
	
	\subsubsection{CaPi}
	
	L'approche CaPi (abbréviation de Cases/pieces) est l'approche la plus naive, elle permet de définir quelle piece peut être placée sur telle case et inversement, quelle case peux avoir telle piece.
	
	Cette approche est l'interaction la plus basique de notre problème. C'est aussi celle-ci qui est utilisée en bruteforce.
	
	\subsubsection{BoCo}
	
	L'approche BoCo (Bordure/Couleur) est bien plus fine : si l'on connait quelle piece est sur telle case, on sait quelle couleur peux se placer sur telle bordure [de la case]. Elle permet d'implémenter un système de mis à jour bien plus performant car ne nombre de couleurs est bien plus petit que le nombre de pièces. 
	
	\begin{exmp}
		Si une couleur disparait, alors tt les pièces ayant cette couleur ne peuvent plus être placés à cette case, par conséquent, les autres bords de la case ont (probablement) des couleurs qui disparaissent aussi (propagation de la disparition).
	\end{exmp}
	
	\subsubsection{Corolles}
	
	Grace aux visions CaPi et BoCo, il est possible de pré-calculer des zones du plateau nommés corolles, ceux-ci contiennent tous les cas possibles dans cette zone donnée.
	
	Le nombre de cas possible étant très important, il est nécessaire de le classer. Les corolles peuvent êtres identifiés grâce à plusieurs critères.
	
	\begin{itemize}
		\item taille du plateau
		\item l'orientation de la corolle
		\item La pièce (et sa rotation) à l'origine de la corolle
		\item La case à l'origine de la corolle
		\item La taille de la corolle
	\end{itemize}
	
	\paragraph{position et orientation des corolles}

	\subsubsection{BoCoDiag}
	
	\section{Resultat}
	
	\section{Manuel d'utilisation}
	
	\section{Manuel Technique}

\end{document}