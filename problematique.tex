% !TeX root = main.tex
\section{Problématique}

	Le but du stage est donc de déterminer si, malgré les dires, on pourrait trouver une méthode de résolution plus efficace que la bruteforce, qui reposerait sur une grande quantité d'information pré-calculées, cette méthode sera appelée smartforce par la suite.

	Ces informations pré-calculées serviront différentes causes, mais deux objectifs principaux peuvent en être explicités :

	\begin{itemize}
		\item les ouvertures
		\item les finales
	\end{itemize}

	Afin de comprendre le principe des ouvertures et des finales, il est important de pouvoir se représenter un arbre de possibilités où chaque branche de l'arbre est une combinaison spécifique.

	\begin{exmp}
		Prenons pour exemple l'arbre des possibilités d'un lancé de monnaie (en supposant que la monnaie ne tombe pas sur la tranche \cite{murray1993probability}). A chaque étape, deux choix s'offrent à nous :
		\begin{itemize}
			\item la pièce tombe sur pile
			\item la pièce tombe sur face
		\end{itemize}

    Supposons que l'on lance la pièce trois fois. On à donc un arbre ressemblant à ça :

    \begin{figure}[H]
    	\begin{center}
    		\includestandalone[mode=image]{graphics/coin_tree}	
    	\end{center}
    \caption{Arbre de possibilités d'un lancer de monnaie}
    \label{fig:coin_tree}
    \end{figure}

    Grâce à cet arbre, à chaque fois qu'une pièce sera lancée jusqu'à trois fois d'affilés, la combinaison (ou chemin) figurera dans l'arbre.
    
    Chaque n\oe ud de l'arbre possède un sous-arbre (si il est pris comme racine), ce sous-arbre peut être vide.
	\end{exmp}
	
	\begin{note}
		Pour l'arbre de résolution d'EternityII, c'est à peu près la même chose mais l'arbre est bien plus grand. Les n\oe uds de celui-ci ne peuvent pas être prédits (on ne connait que les n\oe uds suivants du chemin emprunté).
	\end{note}

	Cet arbre sera représenté par la suite sous cette forme :
	\begin{figure}[H]
	   	\begin{center}
	   		\includestandalone[mode=image]{graphics/triangle}	
	   	\end{center}
	   	
	   	\caption{Représentation simplifiée d'un arbre de possibilité}
	   	\label{fig:arbre}
	\end{figure}
	

	\subsection{Ouvertures}
	
	L'idée est de pré-calculer toutes les débuts jusqu'à un certaine profondeur de l'arbre pour ensuite créer des sous-arbres pouvant être parcourus parallèlement ou pondérer les sous-arbres générés pour les classer (par taille, \dots).
	
	\begin{exmp}
		Dans le cas du lancé de monnaie, je sais que le premier lancer, me donne soit pile soit face. En connaissant cela, je peux séparer l'arbre en deux sous-arbres. Cela m'évite de recalculer la première profondeur
		

		
		\begin{figure}[H]
			\minipage{0.49\textwidth}
			\begin{center}
				\includestandalone{graphics/coin_sub_tree_even}
			\end{center}
			\caption{Sous-arbre de pile}\label{fig:sous-arbre-pile}
			\endminipage\hfill
			\minipage{0.49\textwidth}
			\begin{center}
				\includestandalone{graphics/coin_sub_tree_odd}
			\end{center}
			\caption{Sous-arbre de face}\label{fig:sous-arbre-face}
			\endminipage\hfill
		\end{figure}
	\end{exmp}
	
	\begin{rem}
		L'utilité des ouvertures dans un lancé de pièce (de monnaie) est très discutable, mais elle prends son sens lorsque
		
		\begin{itemize}
			\item le calcul de chaque n\oe ud est gourmand
			\item la quantité de n\oe uds est importante
		\end{itemize}  
	\end{rem}
	
	L'arbre des possibilités dans le cas des ouvertures est représenté comme ceci :
	
	\begin{figure}[H]
		\begin{center}
			\includestandalone[mode=image]{graphics/ouvertures}	
		\end{center}
		
		\caption{Représentation simplifiée des ouvertures}
		\label{fig:ouvertures}
	\end{figure}
		
	Le triangle des possibilités est tronqué car le début a déjà été pré-calculé, c'est comme si l'on avait collé tout les sous-arbre entre eux. Les ouvertures étant stockés sous forme de données, elle sont représentés par une carré.

	\subsection{Finales}

	Les finales permettent de prédire si le chemin emprunté dans l'arbre mène à une solution. Par conséquent, toutes les finales possibles sont pré-calculés et stockés sous forme de données.
	
	Grâce à cela, on peux connaître si le chemin choisi (ou combinaison actuelle) est possible sans avoir à finir le chemin en entier.
	
	Les finales sont représentés comme ceci :
	
	\begin{figure}[H]
		\begin{center}
			\includestandalone[mode=image]{graphics/finales}	
		\end{center}
		
		\caption{Représentation simplifiée des ouvertures}
		\label{fig:finales}
	\end{figure}
	
	\subsection{Objectif}
	
	Grâce à l'aide de ces deux approches. on peux donc diminuer drastiquement la quantité de calcul nécessaire en augmentant la quantité de données stockés. Par ailleurs, en adaptant la méthode de résolution on peux aussi réduire la profondeur totale de l'arbre.
	
	Par conséquent, on réduit non seulement ce qui reste à déterminer, mais aussi la taille de l'arbre dans son ensemble.
	
	\begin{figure}[H]
		\minipage[b]{0.49\textwidth}
		\begin{center}
			\includestandalone[mode=image]{graphics/ouvertures_finales}	
		\end{center}
		\caption{Représentation de la fusion des ouvertures et des finales}
		\label{fig:ouvertures_finales}
		\endminipage\hfill
		\minipage[b]{0.49\textwidth}
		\begin{center}
			\includestandalone[mode=image]{graphics/objectif}	
		\end{center}
		\caption{Représentation simplifiée de l'objectif comprenant les ouvertures, les finales et les données}\label{fig:objectif}
		\endminipage\hfill
	\end{figure}