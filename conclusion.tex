\section{Conclusion}

\subsection{Rétrospective}

EternityII est un sujet qui m'a beaucoup passionné car le fait d'utiliser une grande quantité de données pour vaincre un problème combinatoire m'a paru très intéressante, mais il m'a fallu un certain temps d'adaptation pour assimiler la complexité de certaines approches. Il est difficile de se représenter à quel point c'est complexe. Heureusement, grâce aux données que l'on à trouvé, on peux maintenant savoir que le problème ne peux absolument pas être pris à la légère. Malgré le travail effectué (notamment de débroussaillage) j'aurais aimé pouvoir investir plus de temps dans ce défi.

En un an de TER et deux mois de stage, nous n'avons pas réussi à résoudre le puzzle de taille 8 (ou un parcours total du 7x7), l'objectif initial était le 10x10. Ce qui nous permet de comprendre pourquoi personne n'à pu résoudre le 10x10 en pratiquement 9ans.

Malgré cela, je suis très satisfait du travail que j'ai effectué et de tout ce que j'ai pu apprendre. J'ai essayé de partager ce que j'ai appris et compris durant ce stage.

\subsection{L'avenir d'EternityII}

Les outils présentés et développés par mes soins (grâce à l'aide de bon nombre de personnes) sont destinés à être réutilisés. Même si l'objectif final n'a pas pu être accompli, on se rapproche doucement de la solution. Comme pour le principe de la smartforce, c'est en accumulant des données, de la connaissance et du savoir que l'on pourra un jour résoudre les problèmes de ce monde.