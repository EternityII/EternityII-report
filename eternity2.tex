% !TeX root = main.tex
\section{Eternity II}
	\subsection{Les origines}
	--- [réel]
	
	Eternity II est le fier successeur de Eternity.
	
	La première version sortie en 1999, était composée de 159 pièces de différentes formes, cependant ces formes peuvent être décomposés en forme de triangles équilatéraux (ou leur moitié) qui devaient être placés sur un plateau octogonal. 
	
	\begin{figure}[H]
		\minipage{0.65\textwidth}
		\includegraphics[width=\linewidth]{images/eternity_1.jpg}
		\caption{Eternity I}\label{fig:eternity_1}
		\endminipage\hfill
		\minipage{0.33\textwidth}
		\includegraphics[width=\linewidth]{images/eternity_1_piece.jpg}
		\caption{Forme d'une pièce d'Eternity I} \label{fig:eternity_1_piece}
		\endminipage\hfill
	\end{figure}

	Son point faible se trouvaient dans la disposition de ces pièces sur le plateau : il était possible de pré-calculer des régions, puis de remplir l'espace restant en s'assurant que celui-ci possède une forme adaptée aux pièces restantes \cite{resolutioneternity}.
	De cette façon, le puzzle fut résolu en à peine un an (contrairement aux 3 ans prévus par le créateur), par deux mathématiciens, qui ont ainsi empoché la récompense s'élevant à $1000000\pounds$.
	
	Après cet \enquote{echec}, Christopher Monckton, le créateur d'Eternity \cite{eternity2maker}, décide en 2008 de sortir une deuxième version, bien plus complexe avec à la clé $2000000$\textdollar pour celui qui arriverait à la résoudre au bout de deux ans.
	
	\begin{figure}[H]
		\includegraphics[width=\linewidth]{images/eternity_2.jpg}
		\caption{La boite et les pièces d'Eternity II}
		\label{fig:eternity_2}
	\end{figure}
	
	C'est un puzzle de 16 par 16 qui sort sous le nom d'Eternity II. Ce puzzle est composé de 256 pièces carrés, qui ont chacune 4 faces colorés (ou un demi motif).
	
	Ces pièces peuvent être classés en trois catégories suivant le nombre de faces grises qu'elles possèdent :
	
	\begin{figure}[H]
	   	\minipage{0.32\textwidth}
	   	\includegraphics[width=\linewidth]{images/piece_coin.png}
	   	\caption{\textbf{pièce de coin :} 2 faces grises}\label{fig:piece_coin}
	   	\endminipage\hfill
	   	\minipage{0.32\textwidth}
	   	\includegraphics[width=\linewidth]{images/piece_bord.png}
	   	\caption{\textbf{pièce de bord :} 1 faces grise}\label{fig:piece_bord}
	   	\endminipage\hfill
	   	\minipage{0.32\textwidth}
	   	\includegraphics[width=\linewidth]{images/piece_interieure.png}
	   	\caption{\textbf{pièce d'intérieur :} toutes les faces de couleur}\label{fig:piece_interieure}
	   	\endminipage
	\end{figure}

	Les pièces ne possèdent pas de formes comme dans un puzzle classique. Afin de les faire correspondre l'une avec l'autre, il est nécessaire que les faces adjacentes de chaque pièce voisine soient de la même couleur, dés lors les pièces \enquote{matchent [\autoref{fig:matching}]} .
	
	\begin{figure}[H]
		\centering
		\includegraphics{images/matching_pieces.png}
		\caption{Deux pièces correctement placés (matchés)}\label{fig:matching}
	\end{figure}
	
	Par conséquent, une pièce peut quasiment être placée n'importe où sur le plateau car son placement dépend des couleurs des pièces d'à côté, de plus, les pièces n'ont pas d'orientation prédéterminés (elles peuvent être rotationnées).
	
	Pour résumer, la plupart des pièces peuvent être posés n'importe où sur le plateau à différentes rotation car la position dépends entièrement des pièces adjacentes posés auparavant. 
	
	Enfin, comme leur nom l'indiquent, les pièces de coins sont les seules à pouvoir être posés dans les coins du plateau, c'est aussi valable pour les pièces de bord qui ne peuvent être placés que sur les bords du plateau, ces deux types de pièces ne peuvent pas être ailleurs car leurs faces grises doit \enquote{matcher [\autoref{fig:matching}]} avec les bords du plateau.
	
		--- [résumé]
	
	Eternity II est un jeu sorti en 2008 qui repose sur un principe assez simple, c'est un puzzle de 16 par 16 qu'il faut réassembler. Il est composé de 256 pièces carrés, qui ont, sur chaque arête une couleur donnée. [images tt ca tt ca]. Afin de pouvoir assembler le puzzle, il suffit placer les pièces de façon à ce que les faces adjacentes soient de même couleur. Comme un puzzle classique, il y a des pièces de coin et de bord. Ceux-ci sont reconnaissables car ils possèdent une ou deux arêtes grises. Par contre, la où ca devient complexe, c'est qu'une pièce n'as pas une place prédéterminée (comme dans un puzzle), c'est à dire qu'elle peux se situer n'importe où sur le plateau. 
	
	\newpage
	\subsection{Le défi}
	--- [réel]
	
	Malgré le fait que la récompense à expiré le 31 décembre de l'année 2010, le problème et l'enthousiasme qu'a engendré Eternity II ne c'est pas calmé pour autant (enfin si\dots un peu). Car loin d'être juste un jeu avec une importante cagnotte il recel en son c\oe ur des secrets d'une certaine valeur.
	
	En effet, jusqu'à maintenant, personne n'a réussi à résoudre ce puzzle, même pas effleuré la solution, malgré l'aide de supercalculateurs et de nombreux spécialistes, que ce soient des mathématiciens ou des informaticiens.
	
	Pourquoi ? Car derrière ce jeu anodin se cache l'un des plus grand problème du monde actuel : les problèmes NP-difficiles. Ceux-ci sont fait de telle sorte que même ne connaissant leur structure ou fonctionnement, il est pratiquement impossible d'en déduire un algorithme (moyen de résoudre) afin de trouver la solution. Ce type de problème est communément appliqué dans le chiffrement. Car le meilleur moyen de cacher une aiguille (solution) est de la cacher dans gros paquet d'aiguilles, plus le tas est gros, plus on met de temps à la [l'aiguille] trouver.
	
	\begin{exmp}
		Le nombre de combinaisons possibles pour Eternity II s'élève à $10^{545}$, c'est à dire environ $10^{450}$ fois le nombre d'atomes dans l'univers connu (estimé à au plus $10^{80}$) !!! Ca fait un gros tas d'aiguilles !!
	\end{exmp}
		
	--- [résumé]
	
	A ce jour, personne n'a réussi à résoudre ce puzzle (même grâce à l'aide de supercalculateurs) malgré les différentes stratégies mise en place. Pourquoi ? Car derrière ce jeu anodin se cache l'un des plus grand problème du monde actuel : les problèmes NP-difficiles. Ceux-ci sont fait de tel sorte que même en connaissant leur structure ou fonctionnement, il est pratiquement impossible d'en déduire un algorithme de résolution. L'une des solutions les plus fiables à ce jour est de tester tout les cas possible (qui est évidemment très important).
	
	\begin{exmp}
		Le nombre de combinaisons pour Eternity II s'élève à $10^{545}$, c'est à dire environ $10^{450}$ fois le nombre d'atomes dans l'univers connu (estimé à au plus $10^{80}$) !!!
	\end{exmp}
	
	\subsection{Les II lois d'Eternity II}
	
	--- [réel]
	
	Pour rendre Eternity II complexe et combinatoire, il est nécessaire de respecter les deux lois d'Eternity II :
	
	\begin{law}
		\textbf{Chaque pièce est unique}
		
		L'unicité des pièces est indispensable pour complexifier le problème, car sinon, on peux considérer qu'une pièce peux être placée à plusieurs endroits, en fonction du nombre de \enquote{clones} qu'elle possède. Ce qui réduit grandement l'espace de recherche [de la solution].
	\end{law}

	\begin{law}
		\textbf{La quantité de couleurs et de pièces est finement calculée}
		
		En effet, si l'on augmente le nombre de couleurs, on obtient des couplage uniques : une pièce ne peux être couplée qu'avec une autre pièce (ou dans le meilleur des cas limite le couplage des pièces). A l'inverse, si il n'y a pas assez de couleurs, on obtient des doublons, les pièces ne sont plus uniques, ce qui va à l'encontre de la première loi.
		
		Par ailleurs, certaines couleurs sont exclusives aux pièces de coin et de bord, car ceux-ci étant liés en eux mais seulement sur le périmètre extérieur du plateau il est nécessaire d'ajouter des couleurs supplémentaires tenant compte qu'ils n'ont que 2 ou 3 faces disponibles (le reste étant des faces grises).
	\end{law}
	
	
	--- [résumé]
	
	Pour rendre ce problème combinatoire, il est nécessaire de respecter plusieurs conditions.
	
	\begin{description}
		\item[Chaque pièce est unique] l'unicité des pièces est importante, car si une pièce est en double, cela veux dire que la pièce peux être placée à deux endroits différents (ce qui simplifie le problème)
		\item[Le ratio de nombre de couleurs par nombre de pièces est calculé] [joindre graphique tt ca tt ca] : il faut qu'il y ait assez de couleur pour que chaque pièce soit unique, mais pas assez pour que l'on puisse déterminer les pièces adjacentes
		
		\begin{exmp}
			Supposons qu'il y ait trop de couleurs. Cela veux dire qu'une pièce à peu de voisins ( car chaque pièce est unique, par conséquent les couleurs sont distribués uniformément à travers les pièces), si la pièce à très peu de voisins, je peux déduire des groupement de pièces assez facilement.
			
			Donc je simplifie mon problème.
		\end{exmp}		
	\end{description}
	
	\subsection{Etat de l'art}
	
	Un grand nombre de méthodes ont été mis en place afin de résoudre ce problème.
	
	Il serait trop long de présenter et décrire les différents méthodes mise en place car il requièrent une certaine connaissance dans les domaines auxquels ils sont appliqués. Malgré tout, Les différentes approches seront notés ici à titre informatif.
	
	Pour commencer, il existe plusieurs solveurs graphiques afin de pouvoir résoudre le puzzle manuellement ou assisté par l'ordinateur. Certains d'entre eux permettent même l'import export de la progression actuelle.
	
	\begin{itemize}
		\item E2\_manual : \url{https://sourceforge.net/projects/e2manual/}(anglais)
		\item E2Lab : \url{http://eternityii.free.fr/}(anglais)
	\end{itemize}
	
	Ensuite, il existe un très grand nombre de solveurs bruteforce plus au moins rapides. Certains d'entre eux peuvent agréger plusieurs machines afin d'augmenter la puissance de calcul via le réseau.
	
	Ce problème a été abordé de façon très varié au niveau théorique et applicatif.
	
	Par exemple, Eternity à été adopté sous forme de graphe \cite{patey2010eternity} ou de sous forme de contraintes \cite{benoist2008programmation}.
	
	On note aussi un procédé intéressent de résolution grâce à une approche nommée \textbf{SAT} (satisfaisabilité booléenne) s'appuyant sur la logique propositionnelle \cite{cuvillierresolution} \cite{heule2009solving}.
	
	--- [résumé]
	
	Nombreux sont ceux qui ont essayé de résoudre le problème... plusieurs moyens ont été mis en place, grâce au solveur SAT (solveur de satisfiabilité booléenne dont le fonctionnement est assez complexe pour ne pas être abordé ici), par une approche graphe ou encore par bruteforce.
	
	Il est estimé que actuellement, l'approche la plus performante est la résolution par bruteforce, car c'est elle qui est la plus rapide.