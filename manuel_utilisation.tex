\section{Manuel d'utilisation}

Dans cette partie nous verrons comment utiliser les différents outils développés pendant la durée du stage.

	\subsection{Pré-requis}
	
	Afin de pouvoir manipuler avec aise l'ensemble des outils présentés ici, il est vivement conseillé de connaître le fonctionnement global d'un Git \autocite{git}.
	
	L'ensemble des outils présentés ici sont hébergés sur un dépot privé de l'UM \autocite{gitlab:eternity}.

	\subsection{EternityII}
	
	\textbf{Pré-requis} : Afin de pouvoir utiliser ce logiciel il est nécessaire de pouvoir compiler du code C++11 et utiliser la commande \enquote{\lstinline[columns=fixed,language=bash]{make}}.
	
	Le programme \textbf{EternityII} est le programme principal développé le long de l'année et du stage. Le code source peux être récupéré ici : \url{https://gitlab.info-ufr.univ-montp2.fr/EternityII/EternityII} (dépôt privé).
	
	Le dépot contient deux versions du programme :
	
	\begin{itemize}
		\item La version basique pour la résolution bruteforce
		\item la résolution smartforce
	\end{itemize}

	\subsubsection{Bruteforce}
	
	La version bruteforce se trouve sur la branche v0.2.2-bruteforce \autocite{gitlab:eternity_bruteforce}. L'interaction avec l'application se fait dans le main :
	
	\begin{lstlisting}[language=c++, caption={Parcours bruteforce sur plusieurs tailles du plateau, en affichant tous les résultats des différents parcours}]
int main()
{
    for (int i = 4; i < 8; ++i) {
        ostringstream str;
         // nom du fichier d'entree contenant l'instance
        str << "assets/pieces_" << i << "x" << i << ".txt";
        FileIn file_in(str.str().c_str());
        Jeu jeu = file_in.initJeu(); // initialise le jeu
        // initialise la classe chargee de la resolution
        Generator generator(jeu); 
        // Affiche les resultats pour tous les parcours
        generator.multipleGeneration(); 
    }
}
	\end{lstlisting}
	
	Pour effectuer un parcours spécifique, la dernière ligne doit être changée :
	
	\begin{lstlisting}[language=c++, caption={Parcours bruteforce sur plusieurs tailles du plateau, en affichant tous les résultats du rowscan}]
int main()
{
    for (int i = 4; i < 8; ++i) {
        ostringstream str;
        // nom du fichier d'entree contenant l'instance
        str << "assets/pieces_" << i << "x" << i << ".txt"; 
        FileIn file_in(str.str().c_str());
        Jeu jeu = file_in.initJeu(); // initialise le jeu
        // initialise la classe chargee de la resolution
        Generator generator(jeu); 
        // Affiche les resultats du parcours
        generator.parcoursBruteForce(Generator::PARCOURS_ROW,0); 
    }
}
	\end{lstlisting}
    
	Les différentes constantes pour le parcours sont :
    
	\begin{itemize}
		\item \lstinline[language=c++]|Generator::PARCOURS_ROW|
		\item \lstinline[language=c++]|Generator::PARCOURS_DIAGONAL|
		\item \lstinline[language=c++]|Generator::PARCOURS_SPIRALE_IN|
		\item \lstinline[language=c++]|Generator::PARCOURS_SPIRALE_OUT|
	\end{itemize}
	
	Les fichiers d'instances se trouvent dans le dossier \enquote{assets}.
	
	Le résultat est affiché en Markdown \autocite{markdown}.
	
	\subsubsection{Smartforce}
	
	La résolution smartforce est en cours de développement, malgré le fait que le framework (outil de développement) soit fini. L'implémentation de CaPi est incomplète et renvoie des données non cohérentes.
	
	Le fonctionnement du framework est explicité dans le manuel technique.
	
	\subsection{EternityII--corolle\_generator} 
	
	Le code de ce programme peux être trouvé ici : \url{https://gitlab.info-ufr.univ-montp2.fr/EternityII/EternityII-corolle_generator} (dépôt privé).
	
	Le code source est inspiré du programme de bruteforce.
	
	\textbf{Utilisation}
	
	Compiler le programme en utilisant make (linux).
	
	Le programme accepte deux arguments :
	
	\begin{itemize}
		\item Le fichier d'instance (qui peux être trouvé dans le dossier assets)
		\item Le hamming maximal à générer (2 par défaut).
	\end{itemize}
	
	\begin{exmp}
		La commande \lstinline[language=bash]|$ ./main pieces_10x10.txt 2|
		
		Génèrera les corolles de toutes les zones en hamming 1 et 2 pour un 10.
	\end{exmp}
	
	\textbf{Personnalisation}
	
	Il est malgré tout possible de forcer le programme à ne générer qu'une zone et un hamming spécifique.
	
	En modifiant les lignes (10-14) du fichier main.cpp :
	
	\lstinputlisting[language=c++, firstline=9]{code/corolle.cpp}
	
	en 
	
	\begin{lstlisting}[language=c++]
generator.initGeneration(0,0,1) // x,y,hamming
	\end{lstlisting}
	
	\subsection{EternityII--cardinality\_counter} 
	
	Le code de ce programme peux être trouvé ici : \url{https://gitlab.info-ufr.univ-montp2.fr/EternityII/EternityII-cardinality_counter} (dépôt privé).
	
	Ce programme est en python, il compte le nombre de pièces uniques à chaque position dans la corolle. Il prends deux arguments :
	
	\begin{itemize}
		\item Le fichier corolle.
		\item (optionnel) \enquote{--o} sauvegarde le résultat dans un fichier du même nom avec \enquote{stat\_} préfixé
	\end{itemize}
	
	\begin{exmp}
		\lstinline[language=bash]|$ python main.py corolle.txt -o|
		Traite le ficher et sauvegarde les comptes dans un fichier nommé \lstinline|stat_corolle.txt|
	\end{exmp}
	
	\subsection{EternityII--corolle\_rotator}
	
	Le code de ce programme peux être trouvé ici : \url{https://gitlab.info-ufr.univ-montp2.fr/EternityII/EternityII-corolle_rotator} (dépôt privé).
	
	\textbf{Description}
	
	Génère un fichier de la corolle à la rotation voulue, en fonction du fichier initial.
	
	\textbf{Usage}
	
	\lstinline[language=bash]|$ python main.py file_path [-o output_file, -r rotation]|
	
	\begin{itemize}
		\item \lstinline|file_path| est le chemin vers le fichier de corolle
		\item \lstinline|-o| ou \lstinline|--output| est le chemin vers le fichier généré
		\item \lstinline|-r| ou \lstinline|--rotation| est la rotation souhaitée (entre 0 et 3)
	\end{itemize}
	
	\begin{rem}
		si \lstinline|-o| est spécifié, alors \lstinline|-r| doit l'être aussi.
	\end{rem}
	\begin{rem}
		si \lstinline|-r| n'est pas spécifié, alors il générera les 3 rotation complémentaire au fichier
	\end{rem}
	
	Le fichier de sortie est généré dans le même répertoire que le fichier d'entrée.